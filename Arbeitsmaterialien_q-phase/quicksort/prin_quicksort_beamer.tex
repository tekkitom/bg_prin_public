\documentclass[aspectratio=169]{beamer}
\usepackage{pgf}

\mode<presentation>
{
%   \usetheme{Madrid}
%   \useinnertheme{circles}
  \setbeamertemplate{navigation symbols}{}
%   \usetheme[hideothersubsections, width=2.4cm]{Hannover}
%   \usetheme{Antibes}
%   \usetheme{Montpellier}
  \usetheme{Singapore}
%   \usecolortheme{seahorse}
   \setbeamertemplate{footline}
   {%
%      \begin{beamercolorbox}{section in head/foot}
%       \usebeamercolor{bg}
       \hskip 1em \footnotesize \insertframenumber{} / \inserttotalframenumber%\hskip 41em \includegraphics[height = .5cm]{../../../../bilder/cc_by-nc_eu.png}
% % cc_by-nc_eu.png: 403x141 px, 72dpi, 14.22x4.97 cm, bb=0 0 403 141
%
% % bwslogo_3.png: 476x392 px, 300dpi, 4.03x3.32 cm, bb=0 0 114 94
%
%       %\hskip 5em
%       %       \input{../bilder/cc_by.png}
%       %\includegraphics[height=.5cm]{../../../../bilder/bwslogo_3.png}
%      \end{beamercolorbox}%
   }
  \usepackage{beamerfoils}
}

\usepackage[german]{babel}
\usepackage[utf8]{inputenc}
\usepackage{times}
\usepackage[T1]{fontenc}
\usepackage{eurosym}
\usepackage{graphicx}
\usepackage{amsmath}
\usepackage[siunitx,european]{circuitikz}
\usepackage{ulem}
\usepackage{listings}
%
\lstset{numbers=left, numberstyle=\tiny, stepnumber=2, numbersep=5pt, language = C++, captionpos=b, alsolanguage=XML, basicstyle=\ttfamily}
% \MyLogo{\includegraphics[height=1cm]{../../../../bilder/bwslogo_3.png}}
% % \includegraphics{../../bilder/bwslogo_3.png}
% % bwslogo_3.png: 476x392 px, 300dpi, 4.03x3.32 cm, bb=
%


\only<presentation>{
  \usepackage{hyperref}
}


\title{Quicksort}

% \date{V 0.1.0 - im Aufbau\\ Stand: \today}%\\

\institute[BWS Hofheim]{Brühlwiesenschule, Hofheim}
\author{Thomas Maul}

% \titlegraphic{Für eigene Teile gilt: \includegraphics[height=1cm]{cc_by-nc_eu.png}}

\begin{document}
\begin{frame}<beamer>
  \titlepage
  % \hyperlink{Teil_2}{\beamerbutton{Go part 2}}
\end{frame}
% \AtBeginSection[] % Do nothing for \section*
% {
%   \begin{frame}<beamer>
%     \frametitle{Inhalt}
%     \tableofcontents[currentsection]
%   \end{frame}
% }


\section{Quicksort}

\begin{frame}{Quicksort - Ausgangssituation}
\begin{figure}
  \begin{tikzpicture}
    \foreach \x in {1,...,10}
      \draw (\x,0) +(-.5,-.5) rectangle ++(.5,.5);
      \draw (1,0) node{5};
      \draw (2,0) node{3};
      \draw (3,0) node{9};
      \draw (4,0) node{4};
      \draw (5,0) node{1};
      \draw (6,0) node{8};
      \draw (7,0) node{6};
      \draw (8,0) node{10};
      \draw (9,0) node{2};
      \draw (10,0) node{7};
  \end{tikzpicture}
  \label{quicksortArrayAusgang}
\end{figure}

Ziel: aufsteigend sortiert
\end{frame}

% \begin{figure}
%   \begin{tikzpicture}
%     \foreach \x in {1,...,10}
%       \draw (\x,0) +(-.5,-.5) rectangle ++(.5,.5);
%       \draw (1,0) node{5};
%       \draw (2,0) node{3};
%       \draw (3,0) node{9};
%       \draw (4,0) node{4};
%       \draw (5,0) node{1};
%       \draw (6,0) node{8};
%       \draw (7,0) node{6};
%       \draw (8,0) node{10};
%       \draw (9,0) node{2};
%       \draw (10,0) node{7};
%       \draw [->](10,-1) -- (10,-0.6);
%       \draw (10,-1.3) node{P};
%   \end{tikzpicture}
%   \caption{Array, Ausgangssituation}
%   \label{quicksortArrayPivot_1}
% \end{figure}

\begin{frame}{Algorithmus}
 \begin{itemize}
   \item rekursiver Algorithmus
   \item Pivotelement (p) ist hier rechts
   \item g = soll größer als P sein
   \item k = soll kleiner als P sein
   \item wenn g und k erfüllt $\rightarrow$ tauschen
   \item final: Tausch g und P
 \end{itemize}
\end{frame}


\begin{frame}
\begin{figure}
  \begin{tikzpicture}
    \foreach \x in {1,...,10}
      \draw (\x,0) +(-.5,-.5) rectangle ++(.5,.5);
      \draw (1,0) node{5};
      \draw (2,0) node{3};
      \draw (3,0) node{9};
      \draw (4,0) node{4};
      \draw (5,0) node{1};
      \draw (6,0) node{8};
      \draw (7,0) node{6};
      \draw (8,0) node{10};
      \draw (9,0) node{2};
      \draw (10,0) node{7};
      \draw [->](10,-1) -- (10,-0.6);
      \draw (10,-1.3) node{P};
      \draw [->](1,-1) -- (1,-0.6);
      \draw (1,-1.3) node{g};
      \draw [->](9,-1) -- (9,-0.6);
      \draw (9,-1.3) node{k};
 \end{tikzpicture}
  \caption{Array, Zeiger auf kleineres und größeres Element (relativ zu Pivotelement)}
  \label{quicksortArrayP_k_g_11}
\end{figure}
\end{frame}

\begin{frame}
\begin{figure}
  \begin{tikzpicture}
    \foreach \x in {1,...,10}
      \draw (\x,0) +(-.5,-.5) rectangle ++(.5,.5);
      \draw (1,0) node{5};
      \draw (2,0) node{3};
      \draw (3,0) node{9};
      \draw (4,0) node{4};
      \draw (5,0) node{1};
      \draw (6,0) node{8};
      \draw (7,0) node{6};
      \draw (8,0) node{10};
      \draw (9,0) node{2};
      \draw (10,0) node{7};
      \draw [->](10,-1) -- (10,-0.6);
      \draw (10,-1.3) node{P};
      \draw [->](3,-1) -- (3,-0.6);
      \draw (3,-1.3) node{g};
      \draw [->](9,-1) -- (9,-0.6);
      \draw (9,-1.3) node{k};
 \end{tikzpicture}
  \caption{Array, Zeiger auf Elemente zum Tausch}
  \label{quicksortArrayP_k_g_12}
\end{figure}

\begin{figure}
  \begin{tikzpicture}
    \foreach \x in {1,...,10}
      \draw (\x,0) +(-.5,-.5) rectangle ++(.5,.5);
      \draw (1,0) node{5};
      \draw (2,0) node{3};
      \draw (3,0) node{2};
      \draw (4,0) node{4};
      \draw (5,0) node{1};
      \draw (6,0) node{8};
      \draw (7,0) node{6};
      \draw (8,0) node{10};
      \draw (9,0) node{9};
      \draw (10,0) node{7};
      \draw [->](10,-1) -- (10,-0.6);
      \draw (10,-1.3) node{P};
      \draw [->](7,-1) -- (7,-0.6);
      \draw (7,-1.3) node{g};
      \draw [->](6,-1) -- (6,-0.6);
      \draw (6,-1.3) node{k};
 \end{tikzpicture}
  \caption{Zweites Paar zum Tauschen}
  \label{quicksortArrayP_k_g_13}
\end{figure}

\end{frame}

\begin{frame}
\begin{figure}
  \begin{tikzpicture}
    \foreach \x in {1,...,10}
      \draw (\x,0) +(-.5,-.5) rectangle ++(.5,.5);
      \draw (1,0) node{5};
      \draw (2,0) node{3};
      \draw (3,0) node{2};
      \draw (4,0) node{4};
      \draw (5,0) node{1};
      \draw (6,0) node{6};
      \draw (7,0) node{8};
      \draw (8,0) node{10};
      \draw (9,0) node{9};
      \draw (10,0) node{7};
      \draw [->](10,-1) -- (10,-0.6);
      \draw (10,-1.3) node{P};
      \draw [->](7,-1) -- (7,-0.6);
      \draw (7,-1.3) node{g};
      \draw [->](6,-1) -- (6,-0.6);
      \draw (6,-1.3) node{k};
 \end{tikzpicture}
  \caption{Array, Hilfszeiger haben die Position gewechselt}
  \label{quicksortArrayP_k_g_14}
\end{figure}
\end{frame}

\begin{frame}
\begin{figure}
  \begin{tikzpicture}
    \foreach \x in {1,...,10}
      \draw (\x,0) +(-.5,-.5) rectangle ++(.5,.5);
      \draw (1,0) node{5};
      \draw (2,0) node{3};
      \draw (3,0) node{2};
      \draw (4,0) node{4};
      \draw (5,0) node{1};
      \draw (6,0) node{6};
      \draw (7,0) node{7};
      \draw (8,0) node{10};
      \draw (9,0) node{9};
      \draw (10,0) node{8};
      \draw [->](10,-1) -- (10,-0.6);
      \draw (10,-1.3) node{g};
      \draw [->](7,-1) -- (7,-0.6);
      \draw (7,-1.3) node{P};
 \end{tikzpicture}
  \caption{Tausch \glq gross\grq \ mit Pivotelement}
  \label{quicksortArrayP_k_g_15}
\end{figure}
\end{frame}

\begin{frame}{Runde 2}
\begin{figure}
  \begin{tikzpicture}
    \foreach \x in {1,...,6}
      \draw (\x,0) +(-.5,-.5) rectangle ++(.5,.5);
    \draw (1,0) node{5};
    \draw (2,0) node{3};
    \draw (3,0) node{2};
    \draw (4,0) node{4};
    \draw (5,0) node{1};
    \draw (6,0) node{6};
    \draw (7, -0.5) rectangle (8, 0.5);
    \draw (7.5,0) node{7};
    \draw (8.5, -0.5) rectangle (9.5, 0.5);
    \draw (9,0) node{10};
    \draw (9.5, -0.5) rectangle (10.5, 0.5);
    \draw (10,0) node{9};
    \draw (10.5, -0.5) rectangle (11.5, 0.5);
    \draw (11,0) node{8};
    \draw [->](6,-1) -- (6,-0.6);
    \draw (6,-1.3) node{$\text{P}_{2a}$};
    \draw [->](11,-1) -- (11,-0.6);
    \draw (11,-1.3) node{$\text{P}_{2b}$};
    \draw [->](5,-1) -- (5,-0.6);
    \draw (5,-1.3) node{k};
    \draw [->](1,-1) -- (1,-0.6);
    \draw (1,-1.3) node{g};
    \draw [->](7.5,-1) -- (7.5,-0.6);
    \draw (7.5,-1.3) node{$\text{P}_1$};
 \end{tikzpicture}
 \caption{Aufteilung, $\text{P}_1$ ist alleine, links und rechts neue Bereiche}
  \label{quicksortArrayP_k_g_21}
\end{figure}
\end{frame}

% \begin{figure}
%   \begin{tikzpicture}
%     \foreach \x in {1,...,6}
%       \draw (\x,0) +(-.5,-.5) rectangle ++(.5,.5);
%     \draw (1,0) node{5};
%     \draw (2,0) node{3};
%     \draw (3,0) node{2};
%     \draw (4,0) node{4};
%     \draw (5,0) node{1};
%     \draw (6,0) node{6};
%     \draw (7, -0.5) rectangle (8, 0.5);
%     \draw (7.5,0) node{7};
%     \draw (8.5, -0.5) rectangle (9.5, 0.5);
%     \draw (9,0) node{10};
%     \draw (9.5, -0.5) rectangle (10.5, 0.5);
%     \draw (10,0) node{9};
%     \draw (10.5, -0.5) rectangle (11.5, 0.5);
%     \draw (11,0) node{8};
%     \draw [->](6,-1) -- (6,-0.6);
%     \draw (6,-1.3) node{$\text{P}_{2a}$};
%     \draw [->](11,-1) -- (11,-0.6);
%     \draw (11,-1.3) node{$\text{P}_{2b}$};
%     \draw [->](5,-1) -- (5,-0.6);
%     \draw (5,-1.3) node{k};
%     \draw [->](1,-1) -- (1,-0.6);
%     \draw (1,-1.3) node{g};
%     \draw [->](7.5,-1) -- (7.5,-0.6);
%     \draw (7.5,-1.3) node{$\text{P}_1$};
%  \end{tikzpicture}
%  \caption{Suche links ist abgeschlossen ($5 < 7$)}
%   \label{quicksortArrayP_k_g_22}
% \end{figure}

% \begin{figure}
%   \begin{tikzpicture}
%     \foreach \x in {1,...,6}
%       \draw (\x,0) +(-.5,-.5) rectangle ++(.5,.5);
%     \draw (1,0) node{5};
%     \draw (2,0) node{3};
%     \draw (3,0) node{2};
%     \draw (4,0) node{4};
%     \draw (5,0) node{1};
%     \draw (6,0) node{6};
%     \draw (7, -0.5) rectangle (8, 0.5);
%     \draw (7.5,0) node{7};
%     \draw (8.5, -0.5) rectangle (9.5, 0.5);
%     \draw (9,0) node{8};
%     \draw (9.5, -0.5) rectangle (10.5, 0.5);
%     \draw (10,0) node{9};
%     \draw (10.5, -0.5) rectangle (11.5, 0.5);
%     \draw (11,0) node{10};
%     \draw [->](10,-1) -- (10,-0.6);
%     \draw (6,-1.3) node{$\text{P}_{2a}$};
%     \draw [->](9,-1) -- (9,-0.6);
%     \draw (9,-1.3) node{$\text{P}_{2b}$};
%     \draw [->](10,-1) -- (10,-0.6);
%     \draw (10,-1.3) node{k};
%     \draw [->](11,-1) -- (11,-0.6);
%     \draw (11,-1.3) node{g};
%     \draw [->](7.5,-1) -- (7.5,-0.6);
%     \draw (7.5,-1.3) node{$\text{P}_1$};
%  \end{tikzpicture}
%  \caption{Tausch rechts (10 und 8)}
%   \label{quicksortArrayP_k_g_23}
% \end{figure}
%
% Somit sind 6 und 10 sortiert.
%
% \subsubsection{Runde 3}
\begin{frame}
\begin{figure}
  \begin{tikzpicture}
    \foreach \x in {1,...,5}
      \draw (\x,0) +(-.5,-.5) rectangle ++(.5,.5);
    \draw (1,0) node{5};
    \draw (2,0) node{3};
    \draw (3,0) node{2};
    \draw (4,0) node{4};
    \draw (5,0) node{1};
    \draw (6, -0.5) rectangle (7, 0.5);
    \draw (6.5,0) node{6};
    \draw (7.5, -0.5) rectangle (8.5, 0.5);
    \draw (8,0) node{7};
    \draw (9, -0.5) rectangle (10, 0.5);
    \draw (9.5,0) node{8};
    \draw (10, -0.5) rectangle (11, 0.5);
    \draw (10.5,0) node{9};
    \draw (11, -0.5) rectangle (12, 0.5);
    \draw (11.5,0) node{10};
    \draw [->](6.5,-1) -- (6.5,-0.6);
    \draw (6.5,-1.3) node{$\text{P}_{2a}$};
    \draw [->](9.5,-1) -- (9.5,-0.6);
    \draw (9.5,-1.3) node{$\text{P}_{2b}$};
    \draw [->](5,-1) -- (5,-0.6);
    \draw (5,-1.3) node{$\text{P}_{3a}$};
    \draw [->](11.5,-1) -- (11.5,-0.6);
    \draw (11.5,-1.3) node{$\text{P}_{3b}$};
    \draw [->](4,-1) -- (4,-0.6);
    \draw (4,-1.3) node{k};
    \draw [->](1,-1) -- (1,-0.6);
    \draw (1,-1.3) node{g};
    \draw [->](8,-1) -- (8,-0.6);
    \draw (8,-1.3) node{$\text{P}_1$};
 \end{tikzpicture}
 \caption{Runde 3, suche links (g, k, $\text{P}_{3a}$)}
  \label{quicksortArrayP_k_g_31}
\end{figure}
\end{frame}


\begin{frame}{Runde 4}

  \begin{figure}
  \begin{tikzpicture}
    \foreach \x in {1,...,5}
      \draw (\x,0) +(-.5,-.5) rectangle ++(.5,.5);
    \draw (1,0) node{1};
    \draw (2,0) node{3};
    \draw (3,0) node{2};
    \draw (4,0) node{4};
    \draw (5,0) node{5};
    \draw (6, -0.5) rectangle (7, 0.5);
    \draw (6.5,0) node{6};
    \draw (7.5, -0.5) rectangle (8.5, 0.5);
    \draw (8,0) node{7};
    \draw (9, -0.5) rectangle (10, 0.5);
    \draw (9.5,0) node{8};
    \draw (10, -0.5) rectangle (11, 0.5);
    \draw (10.5,0) node{9};
    \draw (11, -0.5) rectangle (12, 0.5);
    \draw (11.5,0) node{10};
    \draw [->](3,-1) -- (3,-0.6);
    \draw (3,-1.3) node{k};
    \draw [->](1,-1) -- (1,-0.6);
    \draw (1,-1.3) node{g};
    \draw [->](4,-1) -- (4,-0.6);
    \draw (4,-1.3) node{$\text{P}_4$};
 \end{tikzpicture}
 \caption{Runde 4, suche links (g, k, $\text{P}_{4}$)}
  \label{quicksortArrayP_k_g_41}
\end{figure}
\end{frame}

\begin{frame}
\begin{figure}
  \begin{tikzpicture}
    \foreach \x in {1,...,5}
      \draw (\x,0) +(-.5,-.5) rectangle ++(.5,.5);
    \draw (1,0) node{1};
    \draw (2,0) node{3};
    \draw (3,0) node{2};
    \draw (4,0) node{4};
    \draw (5,0) node{5};
    \draw (6, -0.5) rectangle (7, 0.5);
    \draw (6.5,0) node{6};
    \draw (7.5, -0.5) rectangle (8.5, 0.5);
    \draw (8,0) node{7};
    \draw (9, -0.5) rectangle (10, 0.5);
    \draw (9.5,0) node{8};
    \draw (10, -0.5) rectangle (11, 0.5);
    \draw (10.5,0) node{9};
    \draw (11, -0.5) rectangle (12, 0.5);
    \draw (11.5,0) node{10};
    \draw [->](2,-1) -- (2,-0.6);
    \draw (2,-1.3) node{k};
    \draw [->](1,-1) -- (1,-0.6);
    \draw (1,-1.3) node{g};
    \draw [->](3,-1) -- (3,-0.6);
    \draw (3,-1.3) node{$\text{P}_5$};
 \end{tikzpicture}
 \caption{Runde 5, suche links (g, k, $\text{P}_{5}$)}
  \label{quicksortArrayP_k_g_51}
\end{figure}
\end{frame}


  \label{LastPage}
\end{document}
